\begin{hcarentry}{Ruler}
\label{ruler}
\report{Atze Dijkstra}
\status{active development}
\participants{Atze Dijkstra, Arie Middelkoop, Doaitse Swierstra}
\entry{unchanged, unpinged}% done, 09.06.2006, remove 11/2006?
\makeheader

The purpose of the Ruler system is to describe type rules in such a way that a
partial Attribute Grammar implementation, and a pretty printed \LaTeX\ can be
generated from a description of type rules.  The system (currently) is part of
the EHC (Essential Haskell compiler) project~\cref{ehc} and described in a technical
paper, which is also included in the PhD thesis describing the EHC project.
The system is used to describe the type rules of EHC.  The main objectives of
the system are:
\begin{itemize}
\item To keep the implementation and \LaTeX\ rendering of type rules consistent.
\item To allow an incremental specification (necessary for the stepwise description employed by EHC).
\end{itemize}

Using the Ruler language (of the Ruler system) one can specify the structure
of judgements, called judgement schemes.  These schemes are used to `type
check' judgements used in type rules and generate the implementation for type
rules.  A minimal example, where the details required for generation of an
implementation are omitted, is the following:
\begin{verbatim}
scheme expr =
  holes [ | e: Expr, gam: Gam, ty: Ty | ]
  judgespec gam :- e : ty

ruleset expr scheme expr =
  rule app =
    judge A : expr = gam :- a : ty.a
    judge F : expr = gam :- f : (ty.a -> ty)
    -
    judge R : expr = gam :- (f a) : ty
\end{verbatim}
This example introduces a judgement scheme for the specification of type rules
for expressions, and a type rule for applications (as usually defined in
$\lambda$-calculus).

\textbf{New:}
Arie Middelkoop continues with the development of the Ruler system as part
of his Microsoft Research Scholarship PhD grant.
He will investigate the specification of type rules in a partitioned (stepwise an aspectwise)
fashion,
and the incorporation of solving strategies for typing rules.

\FurtherReading
\begin{compactitem}
\item Homepage (Ruler is part of EHC):

\url{http://www.cs.uu.nl/groups/ST/Ehc/WebHome}

From here the mentioned documentation can be downloaded.
\end{compactitem}
\end{hcarentry}
