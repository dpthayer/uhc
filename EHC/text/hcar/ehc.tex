\documentclass{article}

\usepackage{hcar}

\begin{document}

\begin{hcarentry}{EHC, `Essential Haskell' Compiler}
\report{Atze Dijkstra}
\status{active development}
\participants{%
  Jeroen Fokker,
  Doaitse S. Swierstra,
  Arie Middelkoop,
  Luc\'ilia Camar\~ao de Figueiredo,
  Carlos Camar\~ao de Figueiredo
}% optional
\label{ehc}
\makeheader

\paragraph{What is EHC?}
The EHC project provides a Haskell compiler as well as a description of the
compiler which is as understandable as possible so it can be used for
education as well as research.

For its description an Attribute Grammar system (AG) is used as well as other
formalisms allowing compact notation like parser combinators.  For the
description of type rules, and the generation of an AG implementation for
those type rules, we use the Ruler system.
For source code management we use Shuffle, which allows partitioning the system into a sequence of steps and aspects.
(Both Ruler and Shuffle are included in the EHC project).

The EHC project also tackles other issues:
\begin{itemize}
\item
   In order to avoid overwhelming the innocent reader,
   the description of the compiler is organised as a series of
   increasingly complex steps.
   Each step corresponds to a Haskell subset which itself is an extension
   of the previous step.
   The first step starts with the essentials, namely typed lambda
   calculus; the last step corresponds to full Haskell.

\item
   Independent of each step the implementation is organised into a set of aspects.
   Currently the type system and code generation are defined as aspects,
   which can then be left out so the remaining part can be used as a barebones starting point.

\item
   Each combination of step + aspects corresponds to an actual, that is, an executable compiler.
   Each of these compilers is a compiler in its own right.

\item
   The description of the compiler uses code fragments which are
   retrieved from the source code of the compilers.
   In this way the description and source code are kept synchronized.
\end{itemize}

Currently EHC offers experimental implementation of more advanced features like
higher-ranked polymorphism, partial type signatures, and kind polymorphism.
Part of the description of the series of EH compilers is available
as a PhD thesis.

\paragraph{What is EHC's status, what is new?}
\begin{itemize}
\item
   A Haskell frontend plus Prelude has been made, compiled code runs with an interpreter.
   The compiler has an acceptable memory + resource footprint.
   (done by Atze Dijkstra).
\item
   A GRIN (Graph Reduction Intermediate Notation \cite{boquist99phd-optim-lazy}) based backend is available,
   offering global program optimization and code generation to C (done by Jeroen Fokker) as well as LLVM (done by John van Schie).
\item
   Work has started on formalizing EHC's type system; extending our Ruler system will be part of this effort
   (by Luc\'ilia Camar\~ao de Figueiredo, Carlos Camar\~ao de Figueiredo, Arie Middelkoop, Atze Dijkstra).
\item
   The organisation of EHC into aspects, allowing better partial reuse of EHC.
\item
   Though not a direct part of EHC, its supporting tools (AG, Shuffle) are regularly adapted to allow a cleaner EHC code base.
%\item
%   Arie Middelkoop will continue with the development of the Ruler system~\cref{ruler}. 
\end{itemize}

\paragraph{Is EHC used, can I use EHC?}
Yes, but the answer also depends for what purpose.
Although it compiles a Prelude, we have yet to prepare a release of EHC as a Haskell compiler.
Also, the first release will definitively be a alpha release, meant for play and experimentation,
not for compiling real world programs.

EHC is used as a platform for experimentation, see EHC's webpage for various projects related to EHC.
EHC can be downloaded from our svn repository.

\paragraph{What will happen with EHC in the near future?}
We plan to do the following:

\begin{itemize}
\item
  Make the variant for full Haskell available as a Haskell compiler.
  For this we will stabilize the implementation and add proper documentation.
\item
  Rework the type system to have a more formal underpinning.
  Our intent is to use and extend our Ruler system for this.
\end{itemize}

\FurtherReading
\begin{itemize}
\item Homepage:\\
\url{http://www.cs.uu.nl/wiki/Ehc/WebHome}

\item Attribute grammar system:\\
\url{http://www.cs.uu.nl/wiki/HUT/AttributeGrammarSystem}

\item Parser combinators:\\
\url{http://www.cs.uu.nl/wiki/HUT/ParserCombinators}

\item Shuffle:\\
\url{http://www.cs.uu.nl/wiki/Ehc/Shuffle}

\item Ruler:\\
\url{http://www.cs.uu.nl/wiki/Ehc/Ruler}
\end{itemize}

\end{hcarentry}

\end{document}

