\documentclass{article}
\usepackage{a4}
\usepackage[dutch]{babel}
\topmargin=-10mm
\textwidth=150mm
\textheight=250mm
\oddsidemargin=6mm
\evensidemargin=6mm
\parindent=0pt
\parskip=3pt
\pagestyle{empty}

\begin{document}

\paragraph{De functionele programmeertaal Haskell}

Functioneel programmeren is een aantrekkelijk programmeerparadigma:
doordat functies geen neveneffecten hebben op globale variabelen,
is hun gedrag geheel bepaald door hun parameters.
Dit maakt functionele programma's bij uitstek geschikt voor het formuleren
van eigenschappen en het (al dan niet automatisch) uitvoeren van correctheidsbehoudende
transformaties.

De feitelijke standaard functionele programmeertaal is Haskell. Deze taal is
statisch getypeerd (in tegenstelling tot het oudere Lisp),
kent hogere-orde functies (dus functies kunnen zelf ook worden meegegeven als parameter
of worden opgeleverd door andere functies), en 
heeft een lazy semantiek (het evalueren van parameters wordt uitgesteld totdat
het zeker is dat ze voor het eindresultaat nodig zijn).

Hogere-orde functies maken het mogelijk om generieke algoritmen te formuleren voor
het doorlopen van boomachtige datastructuren. In feite kunnen de meeste algoritmen op
bomen worden geformuleerd door voor elk soort knooppunt aan te geven hoe het resultaat
afhangt van de resultaten van recursieve toepassing van het algoritme op de deelbomen.
Van de eigenlijke boom-traversie, en het daadwerkelijk doen van recursieve aanroepen,
kan voor eens en altijd worden geabstraheerd door deze onder te brengen in een
hogere-orde functie, geparametriseerd met de op elk knooppunt uit te voeren functies.

De lazy semantiek maakt het mogelijk om oneindige datastructuren te manipuleren -- zolang
tenminste slechts een eindig deel daarvan geobserveerd wordt.
Sommige op het eerste gezicht circulaire functiedefinities kunnen daardoor verrassenderwijs toch
tot een zinvol resultaat leiden.

Generieke boom-traversal algoritmen in combinatie met lazy semantiek maken het ooit door Knuth
voorgestelde principe van een `attributen-grammatica' tot een in de praktijk zeer
hanteerbaar programmerconcept.

\paragraph{Compilers voor Haskell}

Voor de verwerking van Haskell-programma's is natuurlijk een compiler nodig.
De meest gebruikte, en in feite enige beschikbare compiler is de `Glasgow Haskell Compiler' (GHC).
Deze is zelf ook in Haskell geschreven en is daarmee een duidelijk voorbeeld van
de toepasbaarheid van Haskell in complexe situaties.
Helaas heeft GHC een dermate lange onstaansgeschiedenis dat het erg lastig is geworden
om er nog uitbreidingen aan toe te voegen.

Daarom hebben we een nieuwe compiler gemaakt: de `Utrecht Haskell Compiler' (UHC).
Dit is, net als GHC, een zelf in Haskell geschreven compiler voor Haskell.
Nieuw is echter de architectuur van UHC. 
Het compileerproces in UHC is opgedeeld in vele kleine, onderling zo veel mogelijk onafhankelijke,
transformatie-stappen.
Dit maakt het mogelijk om te experimenteren met eigenschappen van de compiler,
door het anders rangschikken, weglaten of toevoegen van stappen.
Bovendien is elke transformatie consequent opgezet in attributen-grammatica stijl,
wat de opzet van de transformaties helderder en de beschrijving compacter maakt.

\paragraph{Gehele-programma analyse}

Veelal worden compilers ingezet om afzonderlijke modulen van een programma te compileren.
De separaat gecompileerde modulen worden daarna door een linker samengevoegd tot \'e\'en geheel.
Daarmee verliest men echter de mogelijkheid tot programma-optimalisaties waarbij
ook de inter-modulaire aspecten van belang zijn.

Dit project heeft een bijdrage geleverd aan de ontwikkeling van een back-end voor UHC
waarbij zogeheten `gehele-programma analyse' (d.w.z.\ de analyse van inter-modulaire aspecten)
wordt ingezet.
Dit heeft geleid tot nieuwe technieken voor onder andere:
\begin{itemize}
\item De implementie van lazy evaluatie:
   op de plaatsen in het programma waar uitgestelde berekeningen toch
   worden afgedwonen is de aanroep van een evaluatie-functie nodig.
   Door een inter-modulaire analyse van functie-aanroepen
   en een abstracte interpretatie van heap-knopen kan deze functie
   in veel gevallen worden ge-`inlined', waarmee weer nieuwe 
   optimalisatie-kansen aan het licht komen.
\item De implementatie van overloading van functies:
   de gangbare benadering is het meegeven van extra `dictionary' parameters
   aan overloaded functies. 
   Door een gegeneraliseerde vorm van propagatie van constanten kunnen
   deze extra parameters in een inter-modulaire analyse worden weggewerkt,
   met een effici\"enter programma als resultaat.
\end{itemize}
Deze resultaten, alsmede de rol van de architectuur van UHC, zijn gepresenteerd
in een drietal congresbijdragen. Ze zullen bovendien deel uitmaken van een
medio 2011 te verschijnen proefschrift.


\end{document}