\documentclass{article}

\usepackage{a4}

\usepackage[square,numbers,comma,sort&compress]{natbib}
\usepackage{hyperref}
\title{Heap-points-to analysis}

\begin{document}

\section{Introduction}

\subsection{why heap-points-to}

simplification, eval, apply, transformations, inline, specialise

\subsection{how does it look}

store, heap, eval, apply, environment, flow insensitive, call insensitive,
whole program, closed world, unique identifiers, abstract interpretation

- show example of an analysis result. (sum [1..10])

\section{results}

save results, keep results up to date, transformations

\section{Implementation}

retrieving equations, fixpoint

- show example equations

\subsection{Deriving equations}

- show equation derivation

select equations, case statement, artificial identifiers, result variable,
formals, actuals, unknown function application, caf variable

retrieving equations, store, eval, tag to function map, apply, function to tag
map, tag to tag map,

\subsection{fixpoint}

- the actual fixpoint implementation

dependencies, destructive updates, arrays, work list, change sets

\bibliographystyle{abbrvnat}

\bibliography{../papers}

\end{document}
