\begin{hcarentry}[section,updated]{UHC, Utrecht Haskell Compiler}
\report{Atze Dijkstra}%11/09
\status{active development}
\participants{Atze Dijkstra,
 Jeroen Fokker,
 Doaitse Swierstra,
 Arie Middelkoop,
 Luc\'ilia Camar\~ao de Figueiredo,
 Carlos Camar\~ao de Figueiredo,
 Andres L\"oh, Jos\'e Pedro Magalh\~aes,
 Vincent van Oostrum, Clemens Grabmayer, Jan Rochel,
 Tom Lokhorst, Jeroen Leeuwestein, Atze van der Ploeg, Paul van der Ende, Calin Juravle, Levin Fritz}
\label{uhc}
\label{ehc}
\makeheader

\paragraph{UHC, what is new?}
UHC is the Utrecht Haskell Compiler, supporting almost all Haskell98 features and most of Haskell2010, plus
experimental extensions.
After the first release of UHC in spring 2009 we have been working on the next release, which we expect to have available this summer.
Although UHC did start its life as a compiler for research and experimentation,
much of the recent work has focussed on improving and stabilizing UHC for actual use.
The highlights of the next release will be:
\begin{itemize}
\item Support for building libraries with Cabal.
\item A base library sufficient for Haskell98.
\item Support for most of the Haskell2010 language features.
\item A new garbage collector, replacing the Boehm GC we have been using until recently.
\item More stable implementation of both compiler and runtime, with many bugfixes.
\end{itemize}
All of the above is already available for download from the UHC svn repository.

\paragraph{UHC, what do we currently do?}
As part of the UHC project, the following (student) projects and other activities are underway (in arbitrary order):
\begin{itemize}
\item Jan Rochel: ``Realising Optimal Sharing'', based on work by Vincent van Oostrum and Clemens Grabmayer.
\item Tom Lokhorst: type based static analyses.
\item Jeroen Leeuwestein: incrementalization of whole program analysis.
\item Atze van der Ploeg: lazy closures.
\item Paul van der Ende: garbage collection \& LLVM.
\item Arie Middelkoop (\& Luc\'ilia Camar\~ao de Figueiredo): type system formalization and automatic generation from type rules.
\item Jeroen Fokker: GRIN backend, whole program analysis.
\item C\u alin Juravle: base libraries.
\item Levin Fritz: base libraries for Java backend.
\item Andres L\"oh: Cabal support.
\item Jos\'e Pedro Magalh\~aes: generic deriving.
\item Doaitse Swierstra: parser combinator library.
\item Atze Dijkstra: overall architecture, type system, bytecode interpreter backend, garbage collector.
\end{itemize}

Some of the projects are highlighted directly below.

\paragraph{Type based static analysis (Tom Lokhorst)}
We are working on various static optimization transformations on top
of the recently introduced typed core intermediate language. A
particular focus is optimizing code based on the results of a type
based strictness analysis \cite{Holdermans:2010}. We are currently
investigating several approaches to optimizing higher order functions
that are polymorphic in their strictness properties.


\paragraph{Lazy closures (Atze van der Ploeg)}
We are investigating cheaper ways to construct closures by re-using
information already present in frames (incarnation records). In this
scheme a frame may be used by a closure after the frame's function has
ended so we put frames on the heap instead of the stack. If a frame's
function has ended the frame may contain more information than is
necessary for the closures that use it, the garbage collector needs to
be aware of this so that we do not save to much.


\paragraph{Garbage collection \& LLVM (Paul van der Ende)}
We want to extend the LLVM backend of UHC with accurate garbage
collection. The LLVM compiler is known to do various aggressive
transformations that might break static stack descriptors. We will
exploit the existing shadow-stack functionality of the LLVM framework
to connect it with the garbage collection library.

\paragraph{Generic deriving (Jos\'e Pedro Magalh\~aes)}
Recently we wanted to extend the \textbf{deriving} support in UHC to allow 
deriving for other common type classes (such as \textit{Functor} and 
\textit{Typeable}, for example). However, instead of hard-wiring particular 
classes in the compiler, we decided to allow the user to specify how instances 
should be derived for any type class, using simple generic programming 
techniques. Currently we are working on implementing this new feature and 
providing \textbf{deriving} support for a number of useful classes.

\paragraph{Background}

UHC actually is a series of compilers of which the last is UHC, plus
infrastructure for facilitating experimentation and extension:

\begin{itemize}
\item
  The implementation of UHC is organized as a series of
  increasingly complex steps, and (independent of these steps) a set of aspects,
  thus addressing the inherent complexity of a compiler.
  Executable compilers can be generated from combinations of the above.
\item
  The description of the compiler uses code fragments which are
  retrieved from the source code of the compilers, thus
  keeping description and source code synchronized.
\item
  Most of the compiler is described by UUAG, the Utrecht University Attribute Grammar system~\cref{uuag},
  thus providing a more flexible means of tree programming.
\end{itemize}

For more information, see the references provided.

\FurtherReading
\begin{compactitem}
\item UHC Homepage:
\url{http://www.cs.uu.nl/wiki/UHC/WebHome}

\item Attribute grammar system:
\url{http://www.cs.uu.nl/wiki/HUT/AttributeGrammarSystem}

\item Parser combinators:
\url{http://www.cs.uu.nl/wiki/HUT/ParserCombinators}

\item Shuffle:
\url{http://www.cs.uu.nl/wiki/Ehc/Shuffle}

\item Ruler:
\url{http://www.cs.uu.nl/wiki/Ehc/Ruler}

%% @inproceedings{Holdermans:2010,
%% author = {Holdermans, Stefan and Hage, Jurriaan},
%% title = {Making "stricterness" more relevant},
%% booktitle = {PEPM '10: Proceedings of the 2010 ACM SIGPLAN workshop
%% on Partial evaluation and program manipulation},
%% year = {2010},
%% isbn = {978-1-60558-727-1},
%% pages = {121--130},
%% location = {Madrid, Spain},
%% doi = {http://doi.acm.org/10.1145/1706356.1706379},
%% publisher = {ACM},
%% address = {New York, NY, USA},
%% }

\end{compactitem}
\end{hcarentry}
